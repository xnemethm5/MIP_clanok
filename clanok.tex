% Metódy inžinierskej práce

\documentclass[10pt,twoside,slovak,a4paper]{article}

\usepackage[slovak]{babel}
%\usepackage[T1]{fontenc}
\usepackage[IL2]{fontenc} % lepšia sadzba písmena Ľ než v T1
\usepackage[utf8]{inputenc}
\usepackage{graphicx}
\usepackage{url} % príkaz \url na formátovanie URL
\usepackage{hyperref} % odkazy v texte budú aktívne (pri niektorých triedach dokumentov spôsobuje posun textu)

\usepackage{cite}
%\usepackage{times}

\pagestyle{headings}

\title{Inteligentné domácnosti, ich softvér a zabezpečenie\thanks{Semestrálny projekt v predmete Metódy inžinierskej práce, ak. rok 2021/22, vedenie: Ing. Fedor Lehocki PhD.}} % meno a priezvisko vyučujúceho na cvičeniach

\author{Marco Németh\\[2pt]
	{\small Slovenská technická univerzita v Bratislave}\\
	{\small Fakulta informatiky a informačných technológií}\\
	{\small \texttt{xnemethm5@stuba.sk}}
	}

\date{\small 4. November 2021} % upravte



\begin{document}

\maketitle

\begin{abstract}
Žijeme v dobe, kde sa človek snaží implementovať modernizáciu do takmer všetkého na čo si len dokážeme spomenúť, respektíve, snaží sa nám zjednodušiť každodenný život. Mňa konkrétne zaujala téma inteligentných domácností a ich zabezpečenia. V mojej práci by som Vám chcel priblížiť systém IoT (Internet of Things), ktorý je známy ako svet pripojených zariadení, fyzických objektov, ktoré obsahujú senzor a softvérové komponenty. Taktiež by som sa chcel venovať analýze a celkovému zabezpečeniu domácností proti hackerom, data minerom a samozrejme aj voči klasickým zlodejom.
\end{abstract}



\section{Úvod}

Inteligentné domácnosti, ich softvér a zabezpečenie je téma, ktorú som si zvolil ako semestrálny projekt. Mojou úlohou bude priblížiť Vám tento celok, ktorý sa zaoberá, či už samotnými inteligentnými domácnosťami, pojmom IoT(Internet Of Things) a softvérom používaným pri implementovaní inteligentných domácností. 
Moju prácu som rozdelil na tri veľké časti, konkrétne tieto : smart domácnosti, ktoré nájdeme v časti~\ref{domácnosti}, IoT (Internet of Things) nachádzajúce sa v časti~\ref{IoT}  a zabezpečenie, ktoré sa nachádza v časti~\ref{zabezpečenie}. V každej z týchto tém sa budem venovať konkrétnej problematike, ako napríklad v časti~\ref{domácnosti} (smart domácnosti) si bližšie rozoberieme čo vlastne sú inteligentné domácnosti, aké výhody a nevýhody majú, zariadenia, pomocou ktorých dokážu inteligentné domácnosti fungovať a v neposlednom rade celkové zhodnotenie. V časti~\ref{IoT} (Internet of Things) budem sústrediť moju pozornosť na základné zadefinovanie pojmu IoT, princíp fungovania tohto systému, možnosti využitia a implementácia práve v inteligentných domácnostiach. 
Posledná časť~\ref{zabezpečenie} (zabezpečenie) bude venovaná funkcií, výhodam a nevýhodam, druhom a sppôsobom zabezpečenia a samozrejme nástrojom, ktoré realizujú samotné zabezpečenie.


\section{Smart domácnosti}\label{domácnosti}

Inteligentné domácnosti sú ľudské príbytky obohatené o modernú techniku. Zväčša sú čiastočne, môžu však byť aj plne ovládateľné modernými technológiami a dajú sa rôzne využiť~\ref{domacnosti:využitie}. Pri ovládaní nám dopomáha či už hlasový asistent, mobilné zariadenie, laptop, desktop, tablet alebo dalšie ovládače. Z technického pohľadu je inteligentná domácnosť jeden veľký celok, respektíve sieť, do ktorej patrí mnoho rozličných prvkov, ktoré vykonávajú rôzne činnosti. Táto sieť musí fungovať bezchybne, pretože už len pri malej závade môže nastať domino efekt a celá domácnosť z technického pohľadu skolabuje. Má veľa výhod, avšak musíme počítať aj s nevýhodami, ktoré si porovnáme v podsekcií~\ref{domacnosti:plusy}. Inteligentné domácnosti sa rok, čo rok posúvajú rýchlym tempom vpred, stručne ich zhodnotíme~\ref{domacnosti:zhodnotenie}.

\subsection{Funkcie a zariadenia}\label{domacnosti:využitie}

Smart domácnosti majú jedno hlavné využitie a to je zjednodušiť a spríjemniť bývanie v nich. Existuje však mnoho spôsobov ako využívať schopnosti inteligentých domácností. Bývanie sa stáva inteligentným v momente kedy je prepojené so sieťou, je to podmienka, avšak aj minimálna požiadavka oproti všetkému čo môže byť súčasťou tejto siete. Najčastejšie býva zaimplementované do smart domácností veľké množstvo senzorov, či už je to pohybový senzor, senzor teploty, svetla, rýchlosti a kvality vzduchu a podobne.  Po nich nasleduje inteligentné nasvetlovanie, vykurovanie, vetranie. Taktiež nesmieme zabudnúť na inteligentnú bielu a čiernu techniku ako napríklad práčka, sušička, umývačka riadu, chladnička, vysávač, televízia, tablet, mobilné zariadenia a ďalšie. 

\subsection{Klady a zápory}\label{domacnosti:plusy}

Klady a zápory je možné nájsť pri všetkom na čo si len zmyslíme, nie je tomu ináč ani pri inteligentných domácnostiach. Začal by som tým pozitívnym, v prvom rade je mnoho vecí, ktoré musíme v domácnosti robiť manuálne, avšak vďaka smart domácnostiam a automatizácií to už tak nemusí byť. Ako príklad by som uviedol nasvetlovanie a kontrola teploty. Počas horúcich letných dní dokáže byť v miestnosti poriadne horúco, avšak vďáka automatickým vonkajším roletám tomu môžeme jednoducho predísť. Senzor sledujúci teplotu rozpozná výchylku v teplote a vydá impulz, pomocou ktorého sa spustia rolety a tým sa zníži teplota v miestnosti. Na druhej strane tu máme tie menej pozitívne vlastnosti a to konkrétne zápory. Najväčšou nevýhodou inteligentných domácností je závislosť či už od elektrickej energie alebo internetu. Pri výpadku elektriny je celá inteligentná domácnosť znefunkčnená, jednoduchým riešením by bol záložný zdroj elektrickej energie, avšak to nie je bežným riešením. Ďalšou nevýhodou je chybný systém, kde môže ako som už spomínal nastať domino efekt a celá sieť skolabuje.

\subsection{Zhodnotenie}\label{domacnosti:zhodnotenie}

Ako by som zhodnotil inteligentné domácnosti ? Pre mňa ako človeka, ktorého nesmierne fascinujú a zaujímajú moderné technológie je to úžastná téma. Verím v to, že sa smart domácnosti budú rozvíjať do čo raz väčších a zároveň aj komplexnejších celkov. Pretože mať jednoduchší prístup a prehľad o všetkom čo sa v domácnosti deje je veľmi užitočnou vecou. Dokážu sa vďaka tomu zefektívniť celkové náklady na údržbu domácnosti a tým pádom aj znížiť výdavky, ktoré často začínajú na malých sumách, avšak postupom času sú to väčšie a väčšie sumy. Sú však inteligentné domácnosti cenovo náročné ? Áno sú, pretože by som si dovolil tvrdiť, že smart domácnosti sú ešte len v plienkach a rok čo rok budú prístupnejšie a prístupnejšie všetkým cenovým kategóriám.


\section{IoT (Internet of Things)}\label{IoT}

Internet of Thing, skrátene IoT je systém známy ako svet pripojených zariadení, z toho vyplíva, že dokáže spájať rôzne druhy rozdielnych zariadení. Je taktiež definovaný ako koncept tak ako je už spomenuté v zdroji\cite{IoTbasedSmartHome}, ktorý popisuje budúcnosť kde každodenné predmety budú navzájom prepojené a dokážu jeden druhého popísať a rozpoznať. Zariadenia v IoT komunikujú prostredníctvom IP adries bez ľudskej interakcie. Podrobnejšie popísaný princíp si prejdeme v časti~\ref{IoT:princíp}. IoT je systém využiteľný v každom odvetví, na ktoré si len dokážete spomenúť~\ref{IoT:využitie}. V neposlednom rade si zhodnotíme funkčnosť IoT a to v sekcií~\ref{IoTi:funkčnosť}.


\subsection{Princíp fungovania}\label{IoT:princíp}

\subsection{Možnosti využitia}\label{IoT:využitie}

\subsection{Celková funkčnosť a spoľahlivosť}\label{IoTi:funkčnosť}

\section{Zabezpečenie}\label{zabezpečenie}

Zabezpečenie, pri zakomponovaní rôznych zariadení ako napríklad kamera, pohybový senzor, inteligentné zámky a podobne, môžeme získať plnú kontrolu nad domom, aj keď sme mimo neho. Systém, ktorý je prepojený s internetom dokáže upozorniť majiteľa pri nežiadanom vniknutí prostredníctvom aplikácie, do ktorej cez internetu prepošle systém upozornenie. Následne si môže používateľ cez kameru skontrolovať, čo sa v jeho dome deje. Taktiež môže aplikáciou odomknúť dvere alebo ich zamknúť, pretože sa stáva, že ich nezamkneme alebo máme pocit, že sme ich nezamkli. Výhody a nevýhody zabezpečenia inteligentných domácností si priblížime v časti~\ref{zabezpečenie:vyhody}. Existujú rôzne typy a spôsoby zabezpečenia, ktoré sú podrobnejšie popísané v sekcií~\ref{zabezpečenie:druhy}.


Nejaký text a teraz odaz~\ref{zabezpečenie:nástroje} pokračovanie textu

\subsection{Využitie}\label{zabezpečenie:vyhody}

Zabezečenie je dôležitou súčasťou každej domácnosti, či už je inteligentná alebo nie. Moderní zlodeji sú čím ďalej tým vynaliezavejší, a preto musia zabezpečenia byť čoraz komplexnejšie a hlavne spolahlivejšie.

\subsection{Klady a zápory}\label{zabezpečenie:druhy}

\subsection{Využívané zariadenia}\label{zabezpečenie:nástroje}



\section{Záver} \label{zaver}

Zhodnotenie celého článku, čo sme sa dozvedeli...



% týmto sa generuje zoznam literatúry z obsahu súboru literatura.bib podľa toho, na čo sa v článku odkazujete
\bibliography{literatura}
\bibliographystyle{plain} % prípadne alpha, abbrv alebo hociktorý iný
\end{document}

