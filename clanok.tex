% Metódy inžinierskej práce

\documentclass[10pt,twoside,slovak,a4paper]{article}

\usepackage[slovak]{babel}
%\usepackage[T1]{fontenc}
\usepackage[IL2]{fontenc} % lepšia sadzba písmena Ľ než v T1
\usepackage[utf8]{inputenc}
\usepackage{graphicx}
\usepackage{url} % príkaz \url na formátovanie URL
\usepackage{hyperref} % odkazy v texte budú aktívne (pri niektorých triedach dokumentov spôsobuje posun textu)

\usepackage{cite}
%\usepackage{times}

\pagestyle{headings}

\title{Inteligentné domácnosti, ich softvér a zabezpečenie\thanks{Semestrálny projekt v predmete Metódy inžinierskej práce, ak. rok 2021/22, vedenie: Ing. Fedor Lehocki PhD.}} % meno a priezvisko vyučujúceho na cvičeniach

\author{Marco Németh\\[2pt]
	{\small Slovenská technická univerzita v Bratislave}\\
	{\small Fakulta informatiky a informačných technológií}\\
	{\small \texttt{xnemethm5@stuba.sk}}
	}

\date{\small 24. október 2021} % upravte



\begin{document}

\maketitle

\begin{abstract}
\ldots
\end{abstract}



\section{Úvod}

Inteligentné domácnosti, ich softvér a zabezpečenie je téma, ktorú som si zvolil ako semestrálny projekt. Mojou úlohou bude priblížiť Vám tento celok, ktorý sa zaoberá, či už samotnými inteligentnými domácnosťami, pojmom IoT(Internet Of Things) a softvérom používaným pri implementovaní inteligentných domácností. 
Moju prácu som rozdelil na tri veľké časti, konkrétne tieto : smart domácnosti, ktoré nájdeme v časti~\ref{domácnosti}, IoT (Internet of Things) nachádzajúce sa v časti~\ref{IoT}  a zabezpečenie, ktoré sa nachádza v časti~\ref{zabezpečenie}. V každej z týchto tém sa budem venovať konkrétnej problematike, ako napríklad v časti~\ref{domácnosti} (smart domácnosti) si bližšie rozoberieme čo vlastne sú inteligentné domácnosti, aké výhody a nevýhody majú, zariadenia, pomocou ktorých dokážu inteligentné domácnosti fungovať a v neposlednom rade celkové zhodnotenie. V časti~\ref{IoT} (Internet of Things) budem sústrediť moju pozornosť na základné zadefinovanie pojmu IoT, princíp fungovania tohto systému, možnosti využitia a implementácia práve v inteligentných domácnostiach. 
Posledná časť~\ref{zabezpečenie} (zabezpečenie) bude venovaná funkcií, výhodam a nevýhodam, druhom a sppôsobom zabezpečenia a samozrejme nástrojom, ktoré realizujú samotné zabezpečenie.

Uveďte explicitne štruktúru článku. Tu je nejaký príklad.
Základný problém, ktorý bol naznačený v úvode, je podrobnejšie vysvetlený v časti~\ref{nejaka}.
Dôležité súvislosti sú uvedené v častiach~\ref{dolezita} a~\ref{dolezitejsia}.
Záverečné poznámky prináša časť~\ref{zaver}.

\section{Smart domácnosti}\label{domácnosti}
Nejaký text a teraz odaz~\ref{domacnosti:využitie} pokračovanie textu
Nejaký text a teraz odaz~\ref{domacnosti:plusy} pokračovanie textu
Nejaký text a teraz odaz~\ref{domacnosti:zariadenia} pokračovanie textu
Nejaký text a teraz odaz~\ref{domacnosti:zhodnotenie} pokračovanie textu

\subsection{Využitie}\label{domacnosti:využitie}
dasdad
\subsection{Klady a zápory}\label{domacnosti:plusy}

\subsection{Využívané zariadenia}\label{domacnosti:zariadenia}

\subsection{Zhodnotenie}\label{domacnosti:zhodnotenie}


\section{IoT (Internet of Things)}\label{IoT}
Nejaký text a teraz odaz~\ref{IoT:princíp} pokračovanie textu
Nejaký text a teraz odaz~\ref{IoT:využitie} pokračovanie textu
Nejaký text a teraz odaz~\ref{IoT:implementácia} pokračovanie textu
Nejaký text a teraz odaz~\ref{IoTi:funkčnosť} pokračovanie textu

\subsection{Princíp fungovania}\label{IoT:princíp}

\subsection{Možnosti využitia}\label{IoT:využitie}

\subsection{Implementácia}\label{IoT:implementácia}

\subsection{Celková funkčnosť a spoľahlivosť}\label{IoTi:funkčnosť}


\section{Smart domácnosti}\label{zabezpečenie}

Nejaký text a teraz odaz~\ref{zabezpečenie:vyhody} pokračovanie textu
Nejaký text a teraz odaz~\ref{zabezpečenie:druhy} pokračovanie textu
Nejaký text a teraz odaz~\ref{zabezpečenie:nástroje} pokračovanie textu

\subsection{Využitie}\label{zabezpečenie:vyhody}
dasdad
\subsection{Klady a zápory}\label{zabezpečenie:druhy}

\subsection{Využívané zariadenia}\label{zabezpečenie:nástroje}



V modernej dobe, kedy sa všetko automatizuje,to nie je ináč ani pri domácnostiach. Implementujú sa rôzne rozšírenia, ktoré pomáhajú ľuďom pri obsluhovaní a celkovom fungovaní domácnosti, či už to je vykurovanie, osvetlenie, zabezpečenie a podobne. Všetko toto dokážeme využívať vďaka systému IoT (Internet of Things). Teoreticky aj prakticky dokážeme premeniť náš smartfón na plne funkčný ovládač, ktorým kontrolujeme, meníme, sledujeme všetko to, čo zapadá pod inteligentnú domácnosť.  Zabezpečenie, pri zakomponovaní rôznych zariadení ako napríklad kamera, pohybový senzor, inteligentné zámky a podobne, môžeme získať plnú kontrolu nad domom, aj keď sme mimo neho. Systém, ktorý je prepojený s internetom dokáže upozorniť majiteľa pri nežiadanom vniknutí prostredníctvom aplikácie, do ktorej cez internetu prepošle systém upozornenie. Následne si môže používateľ cez kameru skontrolovať, čo sa v jeho dome deje. Taktiež môže aplikáciou odomknúť dvere alebo ich zamknúť, pretože sa stáva, že ich nezamkneme alebo máme pocit, že sme ich nezamkli.


\section{Záver} \label{zaver}




\section{Nejaká časť} \label{nejaka}

Z obr.~\ref{f:rozhod} je všetko jasné. 

\begin{figure*}[tbh]
\centering
%\includegraphics[scale=1.0]{diagram.pdf}
Aj text môže byť prezentovaný ako obrázok. Stane sa z neho označný plávajúci objekt. Po vytvorení diagramu zrušte znak \texttt{\%} pred príkazom \verb|\includegraphics| označte tento riadok ako komentár (tiež pomocou znaku \texttt{\%}).
\caption{Rozhodujúci argument.}
\label{f:rozhod}
\end{figure*}



\section{Iná časť} \label{ina}

Základným problémom je teda\ldots{} Najprv sa pozrieme na nejaké vysvetlenie (časť~\ref{ina:nejake}), a potom na ešte nejaké (časť~\ref{ina:nejake}).\footnote{Niekedy môžete potrebovať aj poznámku pod čiarou.}

Môže sa zdať, že problém vlastne nejestvuje\cite{Coplien:MPD}, ale bolo dokázané, že to tak nie je~\cite{Czarnecki:Staged, Czarnecki:Progress}. Napriek tomu, aj dnes na webe narazíme na všelijaké pochybné názory\cite{PLP-Framework}. Dôležité veci možno \emph{zdôrazniť kurzívou}.


\subsection{Nejaké vysvetlenie} \label{ina:nejake}

Niekedy treba uviesť zoznam:

\begin{itemize}
\item jedna vec
\item druhá vec
	\begin{itemize}
	\item x
	\item y
	\end{itemize}
\end{itemize}

Ten istý zoznam, len číslovaný:

\begin{enumerate}
\item jedna vec
\item druhá vec
	\begin{enumerate}
	\item x
	\item y
	\end{enumerate}
\end{enumerate}


\subsection{Ešte nejaké vysvetlenie} \label{ina:este}

\paragraph{Veľmi dôležitá poznámka.}
Niekedy je potrebné nadpisom označiť odsek. Text pokračuje hneď za nadpisom.



\section{Dôležitá časť} \label{dolezita}




\section{Ešte dôležitejšia časť} \label{dolezitejsia}





%\acknowledgement{Ak niekomu chcete poďakovať\ldots}


% týmto sa generuje zoznam literatúry z obsahu súboru literatura.bib podľa toho, na čo sa v článku odkazujete
\bibliography{literatura}
\bibliographystyle{plain} % prípadne alpha, abbrv alebo hociktorý iný
\end{document}

